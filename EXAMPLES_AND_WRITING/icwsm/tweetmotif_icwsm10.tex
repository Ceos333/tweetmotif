% http://mail.google.com/mail/#search/to%3Akrieger+OR+to%3Adavidahn+OR+(to%3Adavid+AND+to%3Aahn)+OR+from%3Akrieger+OR+from%3Adavidahn/120e6c9667b99ada
\documentclass[letterpaper]{article}
\usepackage{aaai}

\usepackage{times}
\usepackage{helvet}
\usepackage{courier}
\usepackage[usenames]{color}

\pdfinfo{
/Title (FIXME Formatting Instructions for Authors)
/Subject (Proceedings of the AAAI Press Conference)
/Author (AAAI)}
\setcounter{secnumdepth}{0}

% color.sty ... red green blue cyan magenta yellow

\newcommand{\bto}[1]{\textcolor{blue}{\textbf{[#1 --BTO]}}}

%\newcommand{\codenote}[1]{ \textcolor{cyan}{\textbf{[#1]}} }
\newcommand{\codenote}[1]{}

\newcommand{\link}[1]{\texttt{\small{#1}}}

 \begin{document}
% The file aaai.sty is the style file for AAAI Press 
% proceedings, working notes, and technical reports.
%
\title{TweetMotif: Exploratory Search and Topic Summarization for Twitter}
\author{Brendan O'Connor \\ 
Carnegie Mellon University \\ brenocon@gmail.com
\And
Mike Krieger \\ Meebo, Inc. \\ mikekrieger@gmail.com
\And
David Ahn \\ Microsoft, Inc. \\ daviddahn@gmail.com
}
\maketitle
\begin{abstract}
\begin{quote}
We present TweetMotif, an exploratory search application for Twitter.  Unlike traditional approaches to information retrieval, which present a simple list of messages, TweetMotif groups messages by frequent significant terms --- a query set's subtopics --- which facilitate navigation and drilldown through a faceted search interface.  The topic extraction system is based on syntactic filtering, language modeling, near-duplicate detection, and set cover heuristics.  TweetMotif's subtopic groupings make it easy to obtain both an overview and specific examples of what people are saying; we present examples where it can help deflate rumors, uncover scams, summarize sentiment, and track political protests in real-time.  The system also illustrates possibilities for future work in unsupervised linguistic induction from social media text.  A demo of TweetMotif, plus its source code, is available at http://tweetmotif.com.
\end{quote}
\end{abstract}

\section{Introduction and Motivation}

\bto{probably need to cut this intro crap down}

Every day, people around the world broadcast their thoughts to the world as textual messages in various social media.  On Twitter, a recently popular microblogging service, users post millions of very short messages every day.

An exciting question is how to best organize and enable search for this corpus.  Users and analysts might like to see examples of, and better understand, what millions of people are saying about various topics.  Current work has focused on two extremes: (1) showing individual messages, and (2) showing aggregate volume counts.

The most prominent Twitter search systems currently return a flat list of the most recent messages that matched the user query.  This includes Twitter's own search service, \link{search.twitter.com}, as well as recent offerings incorporated into both Google and Microsoft Bing.  This follows the standard approach in web search and traditional information retrieval, where users are often interested in finding a single document that satisfies their information need.  However, characterizing the relevance of microblog messages is an open question; for many information needs, microblog search needs some sort of summarization.

At the other end, there are many efforts that focus on counting the total number of messages matching a term.  The Twitter website prominently features Trending Topics, a list of terms that have been growing in frequency over the last day, week, or month.  ... many others  ..   http://trendistic.com/  .. bla bla bla ... these methods are good to help attain a very rough idea of what topics people are talking about, but do little to provide specifics.  Indeed, Twitter's Trending Topics are often so mysterious that a new website was started to collect explanations for what they are (\link{whatthetrend.com}), and at least one Twitter client adds such explanatory blurbs to its user interface (\link{brizzly.com}).

% more stuff found in email and notes
% http://tweetmeme.com  (categories and attempt at a directory)
% http://twemes.com/  (kinda lame, just hashtags)
% http://twitturly.com/ (just url's)
% 
% http://www.launchtweet.com/  a specialized jobs-on-twitter finding site.



%% hmm more criticism.  [[ explain twitter topics ]] is a great google query
% http://blog.hubspot.com/blog/tabid/6307/bid/4694/Why-Twitter-Hashtags-and-Trending-Topics-Are-Useless-to-Marketers.aspx

Sentiment and opinion analysis approaches ... examples?? ... problematic because sentiment analysis is a difficult, unsolved task in the field of natural language processing ...

We believe a different approach is necessary ... want a high-level summary of a corpus of messages, while being able to view individual examples to gain insight ...

\bto{and a partial screenshot!  but what query? hopefully pick one for a single running example for algorithm exposition.  i'm seriously thinking about \#twitterpornnames, because there's that awesome punchline of the "passwords" topic.  that's the "uncover scams" example i said in the abstract. downside, it's hard to understand on first glance.  also. why not make the screenshot BIG and on the FRONT PAGE!  OUR APPLICATION IS BEAUTIFUL! BOOM!!!}

\bto { oh yeah, and twitterpornnames made it into a dept homeland security warning }


\section{Description of TweetMotif}

Our system, TweetMotif, responds to a user query like a standard search system.  For a query, it retrieves several hundred of the most recent messages that match that query from a simple index; we use the Twitter Search API.

Instead of simply showing the user this entire resultset as a list, TweetMotif extracts a set of topics to group and summarize these messages.  A topic is simultaneously characterized by (1) a textual label, which is a unigram, bigram, or trigram; and (2) a set of messages, whose texts must all contain the label.  The set of topics is chosen to try to satisfy several criteria, which often conflict:

\begin{enumerate}
\item Frequency contrast: Topic label phrases should be frequent in the query subcorpus, but infrequent among general Twitter messages.  This ensures relevance to the query while eliminating overly generic terms.
\item Topic diversity: Topics should be chosen such that their messages and label phrases minimally overlap.  Overlapping topics repetitively fill the same information niche; only one should be used.
%\item Set cover: Topics should be chosen such that their message union includes as much of the query subcorpus as possible, though it is not necessary to i
\item Topic size: A topic that includes too few messages is bad; it is overly specific.
\item Small number of topics: Screen real-estate and user cognitive load are limited resources.
\end{enumerate}

The goal is to provide the user a concise summary of themes and variation in the query subcorpus, then allow the user to navigate to individual topics to see their associated messages, and allow recursive drilldown.

It could be interesting to formulate these desiderata as a constrained optimization problem (e.g. \bto{cite submodular, constr optim}, but in this preliminary work we heuristically fulfill them through several stages of analysis, described as follows.  \bto{need to slip in excuse that all of tweetmotif was built over a 1 month period.  this may fit well in the Examples section.}

\subsection{Step 1: Tokenization and syntactic filtering}

Tokenization is difficult in the social media domain, and, as is often glossed over in academic natural language processing literature, good tokenization is absolutely crucial for overall system performance.  Standard tokenizers, usually designed for newspapers or biomedical publications, perform poorly.  Our tokenizer correctly handles hashtags, @-replies, abbreviations, times of day, and long strings of punctuation, while preserving emoticons and unicode glyphs (e.g. musical notes) as lexical items.  We made no attempt to handle Asian languages or others that require sophisticated word segmentation, but the tokenizer seems to work well for Spanish and other languages with similar word boundary conventions as English; some queries indeed generate multilingual topic sets.

Unigrams are too narrow a unit of analysis; ideally, we want to extract all phrases and subphrases.  In lieu of developing or adapting a part-of-speech tagger, a prerequisite for standard phrase chunking approaches, we use all unigrams, bigrams, and trigrams (from our fine-grained tokenizations) as candidate topic phrases.  We discard unigrams belonging to a small stopword list of function words, and discard all bigrams and trigrams that cross syntactic boundaries: discard if they include certain types of punctuation tokens in certain positions, or if they end with certain right-binding function words like ``the'' and ``of.''  This simple syntactic filtering greatly improves the coherency of extracted n-grams, though they usually seem worse than the results of a typical phrase chunker.


\subsection{Step N: Score and filter topic phrase candidates}

\codenote{lang_model.py, ranking.py}
From this lexicon of n-gram phrases, we filter to phrases that occur at least 3 times, and score them by the likelihood ratio:
\[\frac{Pr(\textrm{ phrase } \ |\ \textrm{ query subcorpus })}
{Pr(\textrm{ phrase } \ |\ \textrm{ general corpus })}
\]

As is usually the case in language modeling, many candidate phrases do not occur in the background corpus, so their probabilities must be estimated with smoothing so the denominator does not blow up the likelihood ratio.  We use a very simple model, Lidstone smoothing:

\[ Pr(\textrm{ phrase }\ |\ \textrm{ corpus }) = \frac
{ \textrm{phrase count in corpus} + \delta }
{ N + \delta n   }
\]

where $\delta$ is the smoothing parameter, set to 0.5, $N$ is the \bto{TODO David what is it??}, and $n$ is the number of types for the phrase size; e.g. number of unique bigrams, if phrase is a bigram.  We did first try even simpler models, MLE and Laplace smoothing, but they gave poor results; since we were satisfied with Lidstone, we did not try more advanced models like Good-Turing, Kneser-Ney, ``stupid backoff,`` etc. \bto{cites}

.. interestingly, hacky MLE beat Lidstone for a while until tweaked syntactic analysis filtering \bto{explain if room}

It is well-known in the information retrieval literature that this  language model comparison approach is analogous to TF/IDF; e.g. \cite{manning_introduction_2008}.  Thus our approach might be vaguely compared to TF/IDF, which estimates document relevance by balancing the frequency of query terms against their frequencies in a background corpus.  Note that the average Twitter message is 11 words long, and words rarely occur more than once in a message; thus, the count of a word is virtually the same as the count of messages it occurs in (DF and TF are the same).  If messages are considered documents, the notion of document TF is not very useful.  Our approach is more like TF for one giant document consisting of the concatenation of all query subcorpus messages.  This too is an odd analogy.  In general, we believe the microblog search problem will require creative formulations of cross-message phonemena beyond current paradigms in IR.

\subsection{Step N: Merge similar topics}

\codenote{ deduper.py }
Every candidate phrase defines a topic, a set of messages that contain that phrase.  Many phrases, however, occur in roughly the same set of messages, thus their topics are repetitive.  This is undesirable.

First, there are easy merges between subsumed n-gram phrases of differing sizes.  Note that each of an n-gram's label-subsumed (n-1)-grams must conversely subsume its message set.  For example, the message set for the bigram topic ``swine flu'' must be a subset or equal to the two unigram topics ``swine'' and ``flu.''  If the ``swine flu'' topic is in fact equal to the ``flu'' topic, then we discard the ``flu'' topic, since ``swine flu'' is strictly better: we can move from ``flu'' to the more descriptively labeled ``swine flu'' without losing any messages.

\bto{ LOL wish there was a way to incorporate http://xkcd.com/574/ }

But more generally, there are more difficult cases when topics roughly overlap; we should to merge topics if their message sets are sufficiently similar.  We use the Jaccard set similarity metric, which measures the size of the intersection, scaled from 0 to 1.  It has a value of 0\% if there are no shared messages, and is 100\% if all messages are shared; i.e, if neither set has messages that are not in the other.  For topic message sets $s_1$ and $s_2$, merge the topics if:

\[ Jacc(s_1,s_2) = \frac{ |s_1 \cap s_2| }{ |s_1 \cup s_2 | } \geq 0.9 
\]
All pairs of topics are compared, and final topics are connected components of the pairwise $Jacc \geq 0.9$ graph (i.e., single-link clustering, such that topics less than 90\% similar may end up merged).  When several topics are merged, only the intersection of messages is included in the new topic.  There is a label choice problem/opportunity: any of the old topics' labels are now legitimate.  Our heuristic solution usually picks longer and higher scoring labels, and sometimes combines short labels into a skip n-gram.

\subsection{Step N: Group near-duplicate messages}

\codenote{ deduper.py }
When we implemented the basic topic system, a message duplication issue was revealed: the same, or nearly the same, textual message may be repeated many times.  People forward (``retweet'') interesting messages such as jokes and news headlines \bto{cite the new boyd et al paper?  kinda lame the wrote a whole paper on retweeting}; and furthermore, a seemingly huge number of bots repeat advertisements, spam, weather reports, news feeds, other people's tweets, and many other types of messages many thousands of times.  It is a waste of space to always show near-duplicates to the search user; therefore we detect clusters of near-duplicates, display them with a single representative and numeric size, and allow them to be optionally viewed.  \bto{awkward}

.. we use metainfo too

The algorithm simply groups messages whose sets of trigrams have a pairwise Jaccard similarity exceeding 65\%.  (Using a trigram message index cuts down on the potentially quadratic runtime.)  This approximates finding a large shared phrase, since usually two messages share several trigrams only when they are overlapping trigrams from a larger shared n-gram.  We experimented with weighting trigrams by their inverse frequency in the general corpus, but this did not seem to improve results.   

This technique seems to reliably find retweets and other forms of repetition; it also naturally groups together spam.


\subsection{Step N: Finalize topics}

We are now left with a ranked list of topics containing messages in near-duplicate clusters.  After eliminating topics that contain only one near-duplicate cluster, the list is cut off to the top 40 topics, and all messages that did not end up in a topic are put in a catch-all ``more...'' topic.

\section{User interface}

\section{Examples}

.. TODO dredge up query logs, though not many people used it besides us

.. twitterpornnames .. unfortunately i just grepped my incomplete tweet scrape and it's not there

.. sentiment analysis example of some sort.  modifiers, not nouns.  completely unsupervised, biyatches!

.. G20 protesters

.. rumors: balloon boy ... damn maybe i lost that screenshot

.. analyze a single person .. from:USERNAME .   (Only works on high-volume users due to search.twitter limitations.  e.g. from:mashable is nice)

.. folk semantics: sandwich

.. note various parts-of-speech

.. text-topic infections: mother's day => sandwich; \#whoremembers => sandwich

.. show some failures.  trending topics actually perform worst; best use case isn't chasing down these hot trends, but rather exploring the space of what exists on twitter.

\section{Related Work}

Topic models; LSA, LSI, LDA ... also, k-means-style document clustering.  Clusty [[god is there anything else to cite by now??]].

All of these approaches suffer from the topic labeling problem. ..  By contrast, we unify the notions of topics and their labels.  This forces a deterministic relationship between a topic's label and the messages under that topic, which ensures maximum transparency to the user.

Another difference with previous topic modeling work is that topic-message relationships and representations are all discrete (boolean).  LSA/LSI is a vector topic model and LDA is a probabilistic topic model; TweetMotif's topic criteria might be formulated as a \emph{discrete topic model}.  Since user interfaces usually communicate discrete information --- e.g., lists of representative words, or the set of documents belonging to a topic --- the results of LSA, LDA, or document clustering usually have to be discretized anyways for a user interface.  Directly formulating discrete topic models may be a useful approach for future work in exploratory document collection analysis.  \bto{or .. ``combinatorial topic model'' was the other term i've been batting around.  both are zero hits on google!}

\section{Acknowledgments}

Withheld for anonymous review.

\bibliography{tweetmotif.bib}
\bibliographystyle{aaai}


\end{document}
